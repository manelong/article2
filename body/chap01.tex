%%% mode: latex
%%% TeX-master: t
%%% End:
\chapter{绪论}
\label{cha:intro}

\section{研究背景与意义}
\label{sec:general intro}
随着移动设备的普及和互联网技术的快速发展,视频内容数据呈现爆炸式增长。视频
由于其直观、生动的表现形式,已成为信息传播和交流的重要媒介。根据中国视听大数据
(CVB)\cite{CVB}统计显示,全国卫视频道体育赛事播出总场次43329场
,其中直播赛事4901场,全国累计收视规模超247.5亿人次,累计收视时长突破66.6
亿小时。面对海量的体育场景视频,传统的人工视频分析技术已无法满足人们的需求,
针对体育场景的智能化视频内容理解技术已成为研究人员关注的重点。

人工智能技术与体育竞技进行深度融合是行业发展的趋势。根据国家体育总局相关报道\cite{国家体育报},
“体育+人工智能”行动将被纳入《“十五五”体育科教发展规划》,将汇聚各方科技力量大力推进
人工智能在体育行业的应用。在体育运动领域中,基于人工智
能的视频分析技术已经在运动员训练辅助、体育赛事分析、赛事转播等多个场景都有丰富的实
际应用并发挥着重要作用。

% 在运动员训练中,利用人体三维重建技术,准确分析运动员的
% 动作准确性,辅助动作矫正;在体育赛事分析中,通过视频理解大模型对体育视频进行智能分
% 析,将运动员的各项数据进行量化和可视化,从而提升赛事分析的智能化水平;在赛
% 事转播中,可以利用计算机视觉和虚拟现实等技术实现智能信息播报和结果预测,增强与线上
% 观众的互动,提升观赛体验。随着相关技术的快速发展,如AI裁判、AI教练、AI自动集锦等新兴人
% 工智能技术正逐步渗透到体育运动的各个环节,推动体育产业的蓬勃发展。

\begin{figure}[!htbp]
    \centering
    \includegraphics[width=0.9\textwidth]{figures/体育时空动作检测应用_1-1.png}
    \caption{体育时空动作检测的应用}\label{Fig1-1}
\end{figure}
时空动作检测作为视频理解领域下的关键技术之一,能在未经剪辑的视频中识别出关注的动作类
别、定位动作发生的起始帧与结束帧,并给出动作主体的空间位置,同时实现时间和空间维
度的定位,已在包括智能安防、自动驾驶、虚拟现实等领域有着广泛的应用。近年来,针对体育
视频的时空动作检测技术得到了广泛关注,由于其可以从海量未剪辑的体育运动视频中,精确地
定位出特定的技术动作视频片段,并识别出运动员的动作类别和运动主体的位置,为运动员训练、
体育赛事分析、高光视频生成等提供有力的数据支持,为诸多下游应用提供了技术基础。

不同于常规的安防场景,体育运动场景下的时空动作检测有着多重挑战。对于体育运动场景
,运动的动作通常复杂多变、运动员运动剧烈、速度快,动作过程中伴随大范围的形变和位移
,且在运动竞技过程中难免存在运动主体相互遮挡的情况。根据相关统计\cite{ART},体育动作场景的
动作内复杂度和动作间复杂度是日常动作的3-8倍,为此,如何提高复杂运动情况下的帧间目标
一致性是体育视频时空动作检测的一大关键。此外,体育运动视频中动作类别繁多,部分技术
动作之间的视觉细微差异也会对时空动作检测带来困难,如足球中的传球和射门、篮
球中的二分球和三分球,其动作特征十分相似的,主要的区别在于动作发生的位置和动作的意图,
这要求模型在理解动作运动特征之外,还需要对全局视频的时空视觉线索和动作间的细节差异进行充分的挖掘,只有
对包括场地信息、运动员信息和专业动作高级语义等信息进行充分关系建模,才能准确检测
体育场景下的动作类型。

综上所述,针对体育运动场景的时空动作检测技术的研究,不仅有具体的实用价值,而且具有重要的
学术研究意义。在应用方面,体育运动场景的时空动作检测技术可以提升体育视频分析的效率和精度
,不仅可以通过对运动人员的动作进行精确的分析,切实地指导训练而提高运动员的竞技能力,
为赛事转播提供智能化的技术支持,提升观众的观赛体验;在学术研究
方面,深入研究体育视频时空动作检测技术,有助于推动视频理解相关技术在处理复杂运动场景视频
的能力。因此,本文针对体育运动场景下的时空动作检测技术展开深入研究。

\section{国内外研究现状}
\label{sec:requirement}
\subsection{时空动作检测研究现状}

时空动作检测是动作识别任务和时序动作检测任务的延伸。动作识别旨在对一段
视频片段中发生的动作进行分类,仅关注这段视频发生了什么动作,属于基础的
视频分类任务;时序动作检测则需要在动作识别的基础上,确定动作在视频中发
生的时间区间,同时实现动作分类和时间维度的定位;而时空动作检测不仅仅需要
知道视频动作的类别和发生区间,还需要在空间维度上对动作目标主体进行定位,
给出动作发生过程中在视频画面中的位置,从而实现时间维度和空间维度的定位。

近年来,随着深度学习技术的快速发展,时空动作检测技术取得了显著进展。现有
的时空动作检测方法根据算法结构范式可以主要分为双阶段方法和单阶段方法
两大类,本节将分别介绍双阶段和单阶段时空动作检测方法的研究现状。

\textbf{(1)双阶段时空动作检测}

双阶段时空动作检测检测方法通常会依赖于额外的目标检测器或区域候选网络(Region Proposal Network,RPN)\cite{fast-r-cnn},预先生成
的候选动作主体的ROI(Region of interest)区域,再通过对ROI区域进行时空特征提取和环节关系建模完成
时空动作检测任务。

如图\ref{1-2}所示,根据所使用的ROI区域的是否存在时间维度,可以将双阶段方法分为帧级和片段级。前者通
常会利用关键帧上的所产生的ROI区域尝试与全局视觉特征进行交互融合,从而提升模
型的环境理解能力;后者则更加关注于如何生成高质量的动作管级别3D候选区域,以
提升模型的动作理解能力。

\begin{figure}[!htbp]
    \centering
    \includegraphics[width=\textwidth]{figures/帧级、片段级时空动作检测_1-2.png}
    \caption{双阶段时空动作检测算法可分为(a)帧级检测方法,(b)片段级检测方法}\label{Fig1-2}
\end{figure}
帧级的双阶段方法受益于目标检测任务的发展,SAHA等人\cite{saha}首次在时空动作检测任务中引入了基于Fast
 R-CNN\cite{fast-r-cnn}的RPN网络来代替传统的无监督区域生成算法,实现了对动作主体的高效定位,
他们将RGB图像和光流图像分别输入到两个独立的RPN网络,以输出检测框和动作
类别得分。ACAM\cite{acam}通过设计一种关系建模模块,对由RPN网络生成的候选区域特征与全局
特征图进行交互建模,提升了模型对环境相关动作的检测能力。MRSN\cite{mrsn}则是基于Vision transformer\cite{vit}
架构,将全局视频特征与局部候选区域特征进行Patch化处理,通过Transformer机制\cite{transformer}进行
运动员特征与全局视觉特征的交互融合,增强了模型的场景理解能力。HIT\cite{hit}额外引入
了人体关键点和手部区域信息,通过融合人体-手部-关键点三重特征,提升了模型对细
粒度动作的识别能力。EVAD\cite{evad}通过设计了基于关键帧的Token dropout机制,来提升
模型的推理效率。

片段级的双阶段方法则是更加关注于如何在进行关系建模之前进行帧间目标关联,
以生成高质量的动作管级别3D候选区域。CFAD\cite{cfad}提出了一种从粗到精获取3D ROI的新范
式,通过粗略模块对长时域信息进行参数化建模,先从视频流中估算出初步的动作管,
随后再利用精细模块,在关键时间戳的引导下,对初步估算的管柱位置进行选择性调整和
细化。TrAD\cite{TrAD}则是通过利用额外的目标跟踪器生成TOI区域获得高质量的动作管候选区域,
为动作分类提供了高质量的空间信息先验。ART\cite{ART}则是参考了目标跟踪中的相关性学习方法,
通过计算帧间的query相似度,设计了query层级的匹配机制,进而生成3D片段级的query,
隐式地实现了帧间目标关联和动作管生成。

\textbf{(2)单阶段时空动作检测}

近年来单阶段目标检测器的快速发展,如YOLO\cite{yolo}、CenterNet\cite{centernet}、Sparse R-CNN\cite{sparse-r-cnn}、DETR\cite{detr}等范式
通过端到端的模型设计,高效地实现了视觉定位任务。时空动作检测方法受此启发,也产生
了很多单阶段时空动作检测方法,直接对视频帧进行时空特征提取和动作检测,无需额外
的候选区域生成步骤。

YOWO\cite{woo}是参考YOLO首次将单阶段目标检测器思想引入时空动作检测任务的方法,其设计了双分支视觉特征
提取网络,分别使用一个2D backbone和一个3D backbone来提取空间和时序特征,并通过信息融
合模块将两者进行融合,最终通过单个检测头实现动作分类和位置信息的回归。MOC\cite{moc}则是参考
CenterNet\cite{centernet}的思想,设计了一种自上而下的基于中心点时空动作检测方法,通过预测输入
片段关键帧的中心点和其它帧中心点的相对偏移量,以实现了对目标运动信息的提取并直接输出片段级
的预测结果。Tuber\cite{tuber}则是首个将DETR范式引入时空动作检
测任务的方法,通过设计时空动作查询机制,利用Transformer的自注意力机制对时空特征进行
建模,实现了对动作类别和位置的直接预测。在此基础上,STAR\cite{star}则是受益于新型的视频预训练模
型Vivit\cite{vivit},并设计了时空解耦的时空动作查询并解耦时空注意力计算模块,在相关基线上取得了
显著提升。STMixer\cite{stmixer}通过设计自适应采样模块和双分支时空特征融合模块,在高效地提取时空特
征的同时,实现了对时空特征的充分融合,高效地实现了单阶段时空动作检测。STDet\cite{stdet}则是的核心
在于使用可学习的管柱查询直接进行时空动作检测,它彻底舍弃了传统方法中繁琐的手
工预设锚点和低效的帧间关联操作,通过直接对可以在更长的窗口内进行全局回归,这
种长时域建模能力使得模型能够更精确地利用长期信息,并显式预测动作的时间边界。

\subsection{体育视频时空动作检测研究现状}

体育场景的视频理解算法研究不仅可以辅助运动员训练,提高观众的观赛体验,带来明显的
经济效益,而且由于体育运动的特殊性,相比于一般的视频分析任务更具挑战性,因此近年来受到了
广泛关注。

Khurram Soomro等人\cite{ucf-sport}最先关注到体育视频研究价值,根据UCF101\cite{ucf101}数据集衍生出了针对体育场景的
时空动作检测数据集UCF Sports,该数据集包含了10类体育运动动作,并提供了每个动作的时空标
注信息,主要研究了基于传统手工特征的体育视频动作识别方法。随着相关研究的深入,越来越多
更具挑战性的体育视频数据集被提出,如Sports-1M\cite{sports-1m}、FineGym\cite{finegym}、SoccerNet\cite{soccernet}等,涵盖了包括篮球
、足球、体操等多种体育运动场景,为体育视频理解算法的研究提供了丰富的数据资源。

当前针对体育视频理解分析的研究多种多样\cite{socceruni,socceragent,soccermaster,matchtime,sportsgpt,smartboard},
如体育视频质量评估、体育视频问答大模型、AI足球裁判等,极大推动了人工智能技术在体育视频理解方面的应用。
LiuZiao等人\cite{smartboard}提出了smartboard视频助理裁判模型,结合大模型对足球比赛进行结合可视分析与体育数据分析,
引入一个大模型智能作为中介,连接“人的意图”与“底层数据/可视化组件”,该系统会自动生成并排列多个可视
化组件,形成一个直观的战术板用来帮助决策。Matchtime\cite{matchtime}则是一个基于解说文本和视频画面信息进行时序对齐的
大模型解说专家,针对解说内容通常滞后于画面信息的问题提出了一种新的时序匹配机制,通过对齐解说文本和视频画面,实现
更加精准的解说内容定位。SportsGPT\cite{sportsgpt}则是一个面向体育视频理解的大模型,主要针对图片视频
的问答场景,通过引入多模态预训练模型和大规模体育视频数据集,实现了对体育视频内容的深度理解和智能问答。

针对体育场景下的时空动作检测任务研究,近些年也取得了显著的进展。
TAAD\cite{TAAD}针对体育视频场景下的时空动作检测任务,基于当前先进的目标跟踪器YOLOv5-DeepSort\cite{yolov5-sort},
并使用预训练的行人重识别基础模型OsNet-x0-25\cite{osnet}作为特征提取器进一步提高对运动员的外观判别能力,
在进行时空动作检测前便实现了对运动员的高质量跟踪,从而提升了模型在复杂多人运动场景下的
动作检测能力。PoSTAL\cite{finesports}则是创新性地引入了BLIP\cite{blip}多模态预训练模型,通过对体育视频中运动员的衣物颜
色和号码等文本信息进行提取和融合,并通过设计的基于提示的动作编码器和动作管解码器直接预
测动作管,提高了模型对细粒度篮球动作类别的区分能力。HHIDet\cite{sportshhi}则是针对篮球场景和网球场景中运
动员交互关系复杂的问题,使用人体检测器获取视频中所有的人体提议,并提出了显式的交互提议生
成和提议间的信息交换机制,以更好地捕捉主体和客体之间的空间几何关系。

\section{存在的问题}
\label{sec:compile}

虽然目前针对体育视频的时空动作检测技术已经取得了一定的进展,但仍然存在一些亟待解决的关键问题:

(1)专业体育技术动作的相似性导致准确检测存在困难。在体育运动场景下,很多技术动作从视觉上看来
差异较小,比如网球场景的扣球和吊球,足球场景的传球和射门等,这些动作在视觉上往往存在较大的相似
性,对于人类是基于一定的专业知识才能较为准确判别此类动作,而现在的时空动作的检测方法基本依赖于
视觉特征进行动作理解,难以充分挖掘动作类别背后的高复杂语义信息,比如动作的意图和规则,这限制了
模型对细粒度体育动作类别的性能。

(2)复杂运动状态的帧间信息关联存在困难。对于体育场景下动作的运动特点,通常存在高速运动、巨大形变、相
互遮挡等复杂情况,现有的时空动作检测方法大多依赖于预训练的目标检测器或RPN网络生成候选区
域在时间维度进行复制,难以应对大位移、强遮挡等情况,甚至会带来噪声干扰。虽然近年来有一些方法尝
试通过目标跟踪等手段进行实例级别帧间关联,但大多难以适应体育场景下复杂的运动情况,导致生成的动
作管质量不高,影响了后续的动作检测性能,且对于遮挡情况,3D ROI由于几何约束的限制,十分容易引入
干扰信息。

(3)运动主体和场景间的关系信息建模困难。体育运动视频中动作类别繁多,部分技术动作之间的视觉细微差异
也会对时空动作检测带来困难,很多动作只有结合场景信息才能进行准确识别,尤其是在多人运动场
景下,运动员与运动员之间、运动员与环境直接的交互关系复杂。现在的方法大多以运动员为中心,与环境
进行注意力计算以实现交互建模,或通过图神经网络等方式对视频中关键元素的RPN特征进行关系建模但是
此类方法仅局限于局部区域的信息交互,难以实现对全局时空视觉信息的充分挖掘。



% (4)体育运动视频的时空特征存在高度耦合。现有的时空动作检测方法大多采用3D CNN或Transformer等网络结构
% 进行视觉特征提取,也有一些多模态大模型能提取到具有丰富语义的多模态特征,但是这里两种方式得到的特征对
% 时序维度和空间维度进行统一建模,导致时序特征和空间特征存在严
% 重信息耦合。而体育运动视频中,时序特征和空间特征的差异性较大,时序特征主要体现在动作的运动轨迹
% 和速度变化上,而空间特征则更多体现在动作发生的具体位置和环境背景上,二者的差异性使得统一建模
% 难以充分挖掘各自的特征信息,限制了模型对时空动作检测能力。

\section{本文主要内容}
\label{sec:checklist}

针对以上时空动作检测技术在体育场景下存在的关键问题,本文进行了以下具体工作:

% (1) 基于时空动作主题引导的时空动作检测方法。针对复杂运动状态下的帧间信息关联问题,本算法提出了
% 通过设计了一种时空动作主题感知模块TAM(Topic Aware Module)对输入视频片段进行全局主题特征压缩和
% 提取,并通过主题特征引导的帧间信息关联模块TIG(Topic Induced Grouping)实现对复杂运动状态下的帧
% 间信息关联。针对体育动作的关系信息建模困难的问题,本文设计了一种自适应特征时空特征采样策略和
% 可变形时空注意力模块,自适应提取视觉特征中的关键特征点,以提升模型对动作主体的特征提取能力。由此实
% 现特征级的帧间信息关联,提升了模型在复杂运动状态下的动作检测能力。

% (2)基于多模态大模型语义增强的体育视频时空动作检测方法。针对细粒度体育动作准确检测存在困难的问题,本算法
% 利用多模态大模型进行语义增强,基于预训练多模态大模型构建外部体育知识库,同时提取样本级的视觉特征和动作实例级的运动特征,
% 提升模型对复杂体育动作的理解能力。此外,本文设计了一种多模态特征检索机制,通过检索外部体育知识库中的多模态特
% 征,并结合双分支时空特征解耦模块,

% (3)

(1)基于多模态知识库增强的时空动作检测算法研究。针对专业体育技术动作的相似性导致准确检测
存在困难的问题,本文设计了一种基于多模态知识库增强的时空动作检测算法,包括设计了一种基于多模态
大模型构建体育知识库的方法,并基于多模态特征检索机制提升模型对专业体育动作的理解能力。

(2)基于时空一致性的时空动作检测算法研究。针对复杂运动状态下的帧间信息关联问题,本算法提出了
通过设计了一种视频主题感知模块TAM(Topic Aware Module)对输入视频片段进行全局主题特征压缩和
提取,并通过主题特征引导的帧间信息关联实现对复杂运动状态下的帧间信息关联。针对体育动作的关系信
息建模困难的问题,本文设计了一种自适应特征时空特征采样策略和可变形时空注意力模块,自适应提取视
觉特征中的关键特征点,以提升模型对动作主体的特征提取能力。由此实现特征级的帧间信息关联,提升了模
型在复杂运动状态下的动作检测能力。

(3)基于多模态特征引导的时空动作检测算法研究。结合前两项工作,本文提出了一种基于多模态特征引
导的帧间一致性建模时空动作检测算法,通过多模态知识库提升模型对专业体育动作的理解能力,且通过帧
间一致性建模提升模型在复杂运动状态下的动作检测能力,从而提升整体的体育视频时空动作检测性能。


本文共分为6章内容,章节内容之间的关系图如图~\ref{Fig1-3}所示。

\begin{figure}[!htbp]
    \centering
    \includegraphics{figures/Fig1-1.png}
    \caption{组织结构图}\label{Fig1-3}
\end{figure}

第一章:绪论。在本章中,首先介绍了体育视频时空动作检测技术的研究背景与意义。
其次,综述了国内外的体育运动场景下时空动作检测领域的研究现状,并分析和总结
了现有相关研究中仍然存在的关键问题。最后针对相关关键问题,阐述了本文的研究
内容,包括基于时空动作主题引导的时空动作检测算法和基于多模特征解耦的体育视
频时空动作检测算法。

第二章:时空动作检测相关理论和技术。本章详细介绍了与本文工作相关的时空动作
检测相关理论和技术,包括时空动作检测的任务定义、基于查询的时空动作检测方法
多模态信息知识库等技术。然后,我们通过对现有体育场景下时空动作检测数据集进
行详细讨论,细致说明体育场景动作检测的特点。最后,介绍了时空动作检测的性能
评价指标。

第三章:基于时空动作主题引导的时空动作检测方法。针对复杂运动状态的帧间空间信息
关联,本章提出了基于时空动作主题特征引导策略,并结合自适应时空动作采样和可变形
时空注意力灵活地进行帧间信息关联,以解决常规ROI几何约束引入噪声的问题。时空动作
主题特征引导策略通过一个时空动作主题特征感知模块对动作主题特征进行提取,并通过
指导时空特征采样来引导特征级别信息关联。自适应时空动作采样和可变形时空注意力提升
了模型对动作主体的特征提取能力,进而提升了复杂运动状态下的时空动作检测性能。最后
,通过对比实验验证了所提方法的优越性和有效性。

第四章:基于多模特征解耦的体育视频时空动作检测方法。针对多人体育动作情况的关系信
息挖掘和体育运动视频的细粒度时空特征建模,本章提出了一种利用多模态大模型提取多模
态特征构建多模态特征库的方法,以提升模型对复杂体育动作的理解能力。随后,为了使多
模态特征可以更好地适配时空动作检测任务,设计了一种双分支时空特征解耦模块,以分别
空间分支通过空间编码器提取关键帧的空间特征,时序分支通过时序编码器提取时序特征,
并分别通过时间对齐和空间对齐模块进行特征对齐,显著提升了模型对细粒度动作的检测能
力。最后,通过大量实验验证了所提方法在体育视频时空动作检测任务中的有效性和优越性。

第五章:

第六章:总结与展望。本章总结了论文的主要内容和创新点,指出了当前工作的不足之处,
并提出了未来研究的方向。