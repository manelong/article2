
%%% Local Variables:
%%% mode: latex
%%% TeX-master: t
%%% End:

\chapter{体育动作时空动作检测相关理论和技术}
\label{cha:command}


\section{引言}
\label{sec:cover}
本章首先明确了时空动作检测的任务定义,说明其相比于动作检测和时序动作检测任务的优势与
挑战,阐释了其在视频中实现分类、时序定位和空间定位的任务目标(第2.1节)。随后,
本章分析了现有视频特征提取方法的常用基础模型,并说明了其各自优缺点(第2.2节)。然后,基
于现有的检测方法范式,本章分析了双阶段检测范式和单阶段检测范式各自的优势,并重点
说明了基于查询机制的动作检测器的理论基础和其在时空动作检测任务中的使用(第2.3节)。最后,
本章对相关领域的基准数据集和评价指标进行了介绍(第2.4节),并具体分析了体育场景下的时空动作检测
的难点(第2.5节),并对本章内容进行了总结(第2.6节)。

\section{时空动作检测任务定义}
\label{sec:font}

从任务目标来看,时空动作检测是动作分类和时序动作检测的进一步延伸。动作检测任务主要关注对
一段指定视频数据中所包含的动作类别进行分类,通常假设视频数据已经经过剪辑处理,确保
视频中仅包含单一动作类别的信息,因此动作检测任务的主要目标是识别视频中所包含的动作类别。
时序动作检测任务则进一步要求模型不仅能够识别视频中的动作类别,还需要对动作发生的时间段进
行定位,即确定动作的起始时间和结束时间。然而,时空动作检测任务在两者的基础上更进一步,通常
处理的是未经剪辑的长视频数据,视频中可能包含多个动作类别,并且这些动作可能在时间和空间上交织
在一起。因此,时空动作检测任务的目标不仅包括动作分类和时序定位,还需要实现对动作在空间维度的
定位出特定的运动动作片段。

\begin{figure}[!htbp]
    \centering
    \includegraphics[width=\textwidth]{figures/动作检测、时序动作检测、时空动作检测任务定义_2-1.png}
    \caption{时空动作检测任务目标:从未剪辑视频中获取完成动作分类、时间定位和空间定位}\label{Fig2-1}
\end{figure}

具体来说,为了统一表达,假设视频序列为 $V = \{I_t\}_{t=1}^{T}$,动作类别集合为 $\mathcal{C}$。对于动作检测来说
动作识别任务目标是判定整段(或已剪辑好的)视频所属的类别,不涉及具体的时间定位和空间定位。
\begin{equation}
    \Phi{(V)} = c_i
\end{equation}
其中 $c_i \in \mathcal{C}$。
对于时序动作检测任务目标是确定输入视频片段中动作发生的时间范围(何时发生)以及动作类别。
\begin{equation}
    \Phi{(V)} = (c_i, t_b, t_e)
\end{equation}
其中 $t_b$ 为开始帧,$t_e$ 为结束帧。
时空动作检测根据图中的定义,该任务需要同时确定动作的类别、时间范围以及每一帧中的空间位置(何处发生)。
\begin{align}
    \Phi{(V)} &= \left( c_i, \{R_t^i\}_{t=t_b}^{t_e} \right) \\
    \{R_t^i\}_{t=t_b}^{t_e} &= \{ (x_{min}, y_{min}, x_{max}, y_{max})_t \mid t \in [t_b, t_e] \}
\end{align}
其中 $c_i \in \mathcal{C}$ 是动作标签,$\{R_t^i\}$ 是从开始时间 $t_b$ 到结束时间 $t_e$ 每一帧图
像 $I_t$ 中对应的边界框或区域集合。

在图\ref{Fig2-1} 具体展示了时空动作检测相比于动作检测和时序动作检测的任务目标的区别。根据任务目标的差异,可以看出时空动作检测
相较于动作检测和时序动作检测具有更高的复杂性和挑战性,需要模型具备更强的时空理解能力和精确的定位能力,
而由于其能够提供更丰富的动作信息,因此在实际应用中具有更广泛的适用性和价值。表\ref{tab:action_tasks} 对比了
三种任务类型及其典型应用场景。

\begin{table}[htbp]
    \centering
    \caption{动作检测任务类型及其适用场景对比}
      \begin{tabular}{ccc}
      \toprule
      任务类型 & 任务目标 & 典型应用场景 \\
      \midrule
      动作识别 & 动作类别 & 视频分类、内容审核 \\
      时序动作检测 & 动作类别+时序定位 & 视频检索、录像回溯 \\
      时空动作检测 & 动作类别+时序定位+空间定位 & 高光视频生成、运动竞技分析 \\
      \bottomrule
      \end{tabular}%
    \label{tab:action_tasks}%
\end{table}

\section{视频特征提取方法}
动作识别、时序动作检测、时空动作检测这三者的核心均依赖于对视频数据中
时空特征的提取与建模。时空特征融合了视频的空间信息(如目标姿态、场景布局)
与时间信息(如运动轨迹、动态演变),为不同任务提供基础表征。尽管任务的目
标各异,这些方法常共享相同的骨干网络,本小节将从 3D CNN 和 Vision 
Transformer 两大主流框架介绍视频特征提取方法的具体原理。 


\section{时空动作检测方法范式}

\subsection{双阶段时空动作检测方法}
控制器设计的详细叙述……

\subsection{单阶段时空动作检测方法}
稳定性分……

\subsection{基于查询的时空动作检测方法}

\section{相关数据集和评价指标}
\subsection{相关数据集}

\subsection{评价指标}

\section{本章小结}
\label{sec:theorem}

本章主要介绍系统与控制理论类论文正文章节的框架结构。在每章的最后,都需要对该章的内容进行小结,不宜太长,建议1/2-2/3页版面较好。主要小结一下本章用什么理论或方法、做了什么事、得到的重要结果或结论。
